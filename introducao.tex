%A new powerful and flexible organization of documents can be obtained by mixing fuzzy and
%possibilistic clustering, in which documents can belong to more than one cluster simultaneously
%with different compatibility degrees with a particular topic. The topics are represented by
%clusters and the clusters are identified by one or more descriptors extracted by a proposed method.
%We aim to investigate whether the descriptors extracted after fuzzy and possibilistic clustering
%improves the flexible organization of documents. Experiments were carried using a collection of
%documents and we evaluated the descriptors ability to capture the essential information of the used
%collection. The results prove that the fuzzy possibilistic clusters descriptors extraction is
%effective and can improve the flexible organization of documents.

O avanço da computação pessoal, em particular a computação móvel, tem proporcionado um gigantesco
aumento da quantidade de dados armazenados pela humanidade ao longo dos anos. A critério de exemplo,
a popular plataforma de rede social Facebook\footnotemark, produz diariamente mais de 25 $terabytes$
de informação \cite{Havens2012}. \citeonline{Huang2015} informa que a tendência com o avanço das
tecnologias, é que tudo seja integrado a internet, de tal modo, que os pesquisadores já chegam a
dizer que os dados são o novo recurso natural do planeta. Ainda segundo \citeonline{Huang2015},
entre as maiores fontes de geração de dados estão os sistemas governamentais, plataformas de mídias
sociais, assim como arquivos armazenados pelas corporações, como por exemplo, formulários médicos,
opiniões de consumidores, relatórios e etc.. Entretanto, \citeonline{Muggleton2006} ressalta que 
todos esses dados excedem os limites humanos para o uso e compreensão.
\footnotetext{\url{https://facebook.com/}}

Diante desse cenário, instituições públicas e privadas estão sobrecarregadas com a tarefa de
processar essa imensa quantidade de informação em bases de dados com documentos não estruturados e
em diversos formatos \cite{Kobayashi2008}. Estes documentos usualmente são de diversos tipos,
como por exemplo, textos, áudios, imagens, vídeos, documentos HTML, assim como também podem estar 
em diferentes idiomas. 

Nesse contexto, diversas pesquisas tem objetivado a proposição ou refinamento de técnicas para
automatização do processo de análise e aquisição de conhecimento útil desse montante de informações
armazenadas. Porém, devido a multi disciplinaridade inerente desse campo de estudo, o mesmo tem
sido estudado pelas comunidades de mineração de dados, aprendizado de máquina e recuperação de
informação.

A mineração de dados é um campo de estudo que vem obtendo rápidos avanços nos últimos anos, e
segundo \citeonline{Aggarwal2012}, isto se deve aos avanços das tecnologias de $hardware$ e
$software$, o qual possibilitou o massivo armazenamento de diferentes tipos de dados, inclusive os
dados textuais. Portanto, como resultado desse aumento na quantidade de documentos disponíveis na
forma textual, existe uma demanda crescente no desenvolvimento e aprimoramento de métodos e
algoritmos que possam efetivamente processar e extrair padrões dos dados de maneira dinâmica e
escalável. 

Por outro lado, enquanto os dados estruturados já possuem mecanismos bem eficientes de armazenamento
e recuperação, os dados textuais são geralmente gerenciados através de mecanismos de buscas para
suprir essa falta de estruturação. Esses mecanismos de busca possibilitam aos usuários uma
conveniente maneira para recuperar informações em coleções textuais através de consultas com palavras
chaves. Desse modo, compete ao campos de estudo da Recuperação de Informação (RI) a tarefa de
explorar, investigar e propor métodos para otimização da eficiência e efetividade de
sistemas de buscas \cite{Baeza2011}. 

Mas segundo \citeonline{Baeza2011}, as pesquisas de recuperação de informação tem focado
tradicionalmente, mais em facilitar o acesso a informação, do que realizar a descoberta de novos
padrões em documentos textuais, o qual se destaca como sendo o objetivo principal da mineração de
textos. A mineração de textos é uma especialização da mineração de dados, que busca incorporar
atividades de estruturação dos documentos em formatos apropriados, facilitando a aplicação dos
tradicionais métodos de extração de padrões da MD, minimizando as perdas durante a conversão do
formato original não estruturado \cite{Nogueira2013}.

Contudo, uma série de características diferenciam os dados textuais de outras formas de dados. O que
por sua vez, afeta o desempenho das clássicas técnicas de mineração de dados.  Dentre essas
características peculiares, destaca-se como os mais importantes o fato de que os dados são esparsos,
e possuem alta dimensionalidade.

%estruturados. Com isso é notória a importância, de organizar de maneira automatizada, esses
%documentos pelos assuntos ao qual se tratam.

Para cumprir a tarefa de extrair informações relevantes de documentos textuais e identificar as
estruturas inerentes aos dados. A mineração de textos emprega uma variedade de técnicas, as quais se
destacam as tarefas de coleta, pré-processamento, agrupamento textual e seleção de termos
descritores para o agrupamento. 

Os métodos de agrupamento podem ser separados então pela lógica matemática utilizada, que pode ser a
lógica clássica ou a lógica fuzzy. Na lógica clássica, após o agrupamento, cada documento só poderá
pertencer a um grupo, enquanto na lógica fuzzy, a pertinência do documento será distribuída entre os
grupos.  Se analisarmos a diversidade de conteúdo em documentos textuais, é trivial notar que
frequentemente um texto aborda um ou mais temas. Com isso é evidente a necessidade de desenvolver-se
técnicas para organizar de maneira flexível os documentos. Percebe-se então, que os métodos de
agrupamento fuzzy, se mostram coerentes com a realidade multi temática dos documentos textuais. Por
sua vez, o método FCM(fuzzy c means), que é uma adaptação do clássico k means, se propõe a
identificar e separar uma coleção de documentos em grupos, respeitando a lógica multi valorada,
permitindo assim que um documento pertença a um ou mais grupos. No entanto o FCM possui algumas
falhas conhecidas, o que motivou a pesquisa e desenvolvimento de métodos alternativos e baseados no
FCM, com o propósito de sanar estes problemas.  Este é o caso dos métodos PCM(Possibilístico C
Means) e PFCM(Possibilístico C Means).  Para então avaliarmos corretamente o resultado do
agrupamento e a qualidade da organização flexível de documentos, é preciso extrair corretamente os
descritores dos grupos obtidos, levando em consideração a relevância de determinado termo para cada
grupo. Com isso temos um cenário no qual é preciso combinar métodos de agrupamento fuzzy com métodos
de extração de descritores, para obtermos uma bom resultado no processo de organização dos
documentos. A investigação e refinamento dessa combinação de métodos, foi a motivação do presente
trabalho. Como resultado desse trabalho foi, proposto extender os experimentos referentes a
organização flexível de documentos, utilizando novos métodos de agrupamento fuzzy existentes na
literatura, como o PCM e o PFCM. Assim como também foi proposto os métodos de extração de
descritores: i) Mixed-PFDCL ({ \it Mixed - Possibilistic Fuzzy Descriptor Comes Last\/ }), que se
utiliza da abordagem híbrida do algoritmo PFCM, misturando assim descritores fuzzy e
possibilísticos. ii) MixedW-PFDCL ({ \it Mixed Weight - Possibilistic Fuzzy Descriptor Comes Last\/
}), onde além de misturar descritores fuzzy e possibilístico, leva em consideração os parâmetros de
ponderação do método PFCM.  Além dos métodos de extração de descritores, foi conduzido um estudo dos
impactos de se utilizar o algoritmo PCM, no método de agrupamento hierárquico HFCM, o que resultou
no método de agrupamento hierárquico HPCM ({ \it Hierarchical Possibilistic C Means\/ }).
% FALAR SOBRE OS MÈTODOS FUZZY UTILIZADOS, FCM, PFCM, PCM, HFCM, HPCM
% FALAR SOBRE A EXTRAÇÂO DE DESCRITORES
% FALAR SOBRE A MISTURA REALIZADA NA EXTRAÇÃO DE DESCRITORES
