%A new powerful and flexible organization of documents can be obtained by mixing fuzzy and
%possibilistic clustering, in which documents can belong to more than one cluster simultaneously
%with different compatibility degrees with a particular topic. The topics are represented by
%clusters and the clusters are identified by one or more descriptors extracted by a proposed method.
%We aim to investigate whether the descriptors extracted after fuzzy and possibilistic clustering
%improves the flexible organization of documents. Experiments were carried using a collection of
%documents and we evaluated the descriptors ability to capture the essential information of the used
%collection. The results prove that the fuzzy possibilistic clusters descriptors extraction is
%effective and can improve the flexible organization of documents.

O avanço da computação pessoal, em particular a computação móvel, tem proporcionado um gigantesco
aumento da quantidade de dados armazenados pela humanidade ao longo dos anos. A critério de exemplo,
a popular plataforma de rede social Facebook\footnotemark, produz diariamente mais de 25 $terabytes$
de informação \cite{Havens2012}. De acordo com \citeonline{Huang2015}, a tendência com o avanço das
tecnologias, é que tudo seja integrado a internet, de tal modo que os pesquisadores já chegam a
dizer que os dados são o novo recurso natural do planeta. Ainda segundo os autores,
entre as maiores fontes de geração de dados estão os sistemas governamentais, plataformas de mídias
sociais, assim como arquivos armazenados pelas corporações, como por exemplo, formulários médicos,
opiniões de consumidores, relatórios e etc.. Entretanto, \citeonline{Muggleton2006} ressalta que 
todos esses dados excedem os limites humanos para o uso e compreensão destes.
\footnotetext{\url{https://facebook.com/}}

Diante desse cenário, instituições públicas e privadas estão sobrecarregadas com a tarefa de
processar essa imensa quantidade de informação em bases de dados com documentos não estruturados e
em diversos formatos \cite{Kobayashi2008}. Estes documentos usualmente são de diversos tipos,
como por exemplo, textos, áudios, imagens, vídeos, documentos HTML, podendo estar, inclusive, em diferentes idiomas. 

Nesse contexto, diversas pesquisas tem objetivado a proposição ou refinamento de técnicas para
automatização do processo de análise e aquisição de conhecimento útil desse montante de informações
armazenadas. Porém, devido a multi disciplinaridade inerente desse campo de estudo, o mesmo tem
sido estudado pelas comunidades de mineração de dados, aprendizado de máquina e recuperação de
informação.

A Mineração de Dados (MD) é um campo de estudo que vem obtendo rápidos avanços nos últimos anos, e
segundo \citeonline{Aggarwal2012}, isto se deve aos avanços das tecnologias de $hardware$ e
$software$, o qual possibilitou o massivo armazenamento de diferentes tipos de dados, inclusive os
dados textuais. Portanto, como resultado desse aumento na quantidade de documentos disponíveis na
forma textual, existe uma demanda crescente no desenvolvimento e aprimoramento de métodos e
algoritmos que possam efetivamente processar e extrair padrões dos dados de maneira dinâmica e
escalável. 

Por outro lado, enquanto os dados estruturados já possuem mecanismos bem eficientes de armazenamento
e recuperação, os dados textuais são geralmente gerenciados através de mecanismos de buscas para
suprir essa falta de estruturação. Esses mecanismos de busca possibilitam aos usuários uma
conveniente maneira para recuperar informações em coleções textuais através de consultas com palavras
chaves. Desse modo, compete ao campo de estudo da Recuperação de Informação (RI) a tarefa de
explorar, investigar e propor métodos para otimização da eficiência e efetividade de
sistemas de buscas \cite{Baeza2011}. 

Mas segundo \citeonline{Baeza2011}, as pesquisas de recuperação de informação tem focado
tradicionalmente, em formas de facilitar o acesso à informação, do que realizar a descoberta de novos
padrões em documentos textuais, o qual se destaca como sendo o objetivo principal da mineração de
textos. A mineração de textos, por sua vez, é uma especialização da mineração de dados, que busca
incorporar atividades de estruturação dos documentos em formatos apropriados, facilitando a
aplicação dos tradicionais métodos de extração de padrões da MD, minimizando as perdas durante a
conversão do formato original não estruturado \cite{Nogueira2013}.

Contudo, uma série de características diferenciam os documentos textuais de outras formas de dados.
O que por sua vez, afeta o desempenho das clássicas técnicas da MD.  Dentre essas características
peculiares, destacam-se como mais importantes os fatos de que os dados são esparsos e possuem alta
dimensionalidade. Por exemplo, uma coleção de documentos (corpus) pode conter 100.000 palavras (termos), enquanto um único documento desse corpus pode conter somente algumas centenas de palavras \cite{Aggarwal2012}.  Essa discrepância,
tem implicações diretas em várias técnicas de identificação de padrões, e especialmente no
agrupamento textual, que deriva de clássicas técnicas de agrupamento da mineração de dados,
aplicados à conjuntos de baixa dimensionalidade. 
%estruturados. Com isso é notória a importância, de organizar de maneira automatizada, esses
%documentos pelos assuntos ao qual se tratam.

Para cumprir a tarefa de extrair informações relevantes de documentos textuais e identificar as
estruturas inerentes aos mesmos. A mineração de textos emprega uma variedade de técnicas, as quais se
destacam aquelas usualmente desenvolvidas para efetuar tarefas de coleta, pré-processamento, agrupamento textual e seleção de termos descritores para o agrupamento. 

O agrupamento pode, de maneira geral, ser definido como a tarefa de agrupar uma coleção de objetos,
de acordo algum critério de similaridade. É possível distinguir os tipos de agrupamento em
função da lógica empregada por eles. Com isso, tem-se os algoritmos que derivam da lógica clássica ou
da lógica fuzzy. Na lógica clássica, após a conclusão do agrupamento, cada elemento só pertence à
apenas um grupo, enquanto que na lógica fuzzy, a pertinência do elemento será distribuída entre os
grupos. 

Ao se analisar a diversidade de conteúdo em dados textuais, é trivial notar que frequentemente
um texto aborda um ou mais temas. O que implica que o agrupamento clássico, ao atribuir um objeto a
apenas um grupo, não irá representar bem a imprecisão e incerteza natural dos documentos. 

Deste modo, os métodos de agrupamento derivados da lógica fuzzy se mostram mais capacitados para
lidar com essa imprecisão e incerteza da realidade multi temática dos documentos textuais. Assim
sendo, uma organização flexível de documentos pode ser definida como o processo que compreende a
estruturação dos dados, a adição de flexibilidade proporcionada pelo agrupamento fuzzy, a extração
de descritores dos grupos de maneira flexível e a recuperação de informação através de um Sistema de Recuperação de Informação (SRI).

Ao se observar o processo de organização flexível de documentos, percebe-se que o mesmo abrange
várias etapas, cada uma delas com suas particularidades. No entanto, apesar da importância
desempenhada por cada etapa do processo, o agrupamento em si pode ser visto como uma das peças
chaves, pois ele é diretamente responsável por organizar os documentos de acordo com as suas
similaridades. Adicionalmente, é preciso desconsiderar ou reduzir a influência de
documentos ruidosos, que destoam do restante da coleção nos grupos finais.

Os algoritmos Fuzzy C-Means (FCM) \cite{Bezdek1984}, que deriva do clássico K-Means
\cite{Macqueen1967}, e o Possibilistic C-Means (PCM)
\cite{Krishnapuram1993}, são exemplos de métodos de agrupamento capazes de organizar de maneira
automatizada uma coleção de documentos em um conjunto de grupos
\cite{Mei2014,Tjhi2009,Prade2008,Saracoglu2008}. Ambos distribuem os documentos de uma coleção
textual em um conjunto de grupos, de modo que cada documento possa pertencer a diferentes grupos com
diferentes graus de pertinência, considerando assim a flexibilidade necessária para tratar a
imprecisão e incerteza do processo.

No entanto, o FCM apresenta alguns resultados indesejados, diante da presença de dados ruidosos na
coleção. Em se tratando de coleções textuais, um dado ruidoso pode ser considerado como um
documento que possua uma temática bastante diferente dos demais documentos da coleção. Com o
objetivo de atribuir valores de pertinências mais realísticos aos elementos a serem agrupados e
penalizar com baixas pertinências os elementos ruidosos, o algoritmo PCM foi proposto. Porém, o PCM
é muito sensível à inicialização, o que pode resultar em grupos coincidentes, onde não há uma
separação muito bem definida dos elementos.

Visando contemplar os benefícios de ambos os métodos, foi proposto o método de agrupamento Possibilistic Fuzzy C-Means
(PFCM) \cite{Pal2005}, como uma versão híbrida dos algoritmos FCM e o PCM,
objetivando adicionar robustez à tarefa de agrupamento.

Após o agrupamento dos documentos, é necessário realizar a extração dos termos que melhor descrevem
os grupos. Para realização dessa tarefa, tem-se alguns métodos na literatura do tipo DCF ({\it
Description Comes First}), que realizam a extração de maneira embutida no processo de agrupamento.
Porém, essa abordagem torna o processo de extração de descritores dependente do algoritmo de
agrupamento. Com o propósito de contornar esse cenário, foi proposto em \citeonline{Nogueira2013} o
método {\it Soft Organization - Fuzzy Description Comes Last} (SoftO-FDCL) \cite{Nogueira2013}, o qual extrai os termos descritores após a etapa de agrupamento de maneira independente do algoritmo de agrupamento utilizado. Permitindo avaliar diretamente os impactos dos métodos de agrupamento, na
extração de descritores e por consequência na qualidade da organização flexível de documentos.

Entretanto, o método SoftO-FDCL foi pensado inicialmente para interpretar as pertinências produzidas
na partição do FCM, que difere da partição resultante produzida pelo PCM. A principal contribuição
do PCM foi uma alteração no modo de atribuição da pertinência de uma elemento a um grupo, o que
impacta diretamente na partição dos grupos resultantes. 

Diante deste contexto, e tendo em vista o crescente aumento de informações produzidas além da capacidade humana de
analisar. Com a demanda crescente no desenvolvimento e aprimoramento das técnicas de extração e
identificação de conhecimento útil em dados textuais, bem como a necessidade de se organizar esses dados de
maneira flexível, tratando a imprecisão e incerteza natural desses dados e considerando as
particularidades existentes nos métodos de agrupamento, foi formulada a seguinte hipótese para o desenvolvimento desse trabalho:

\begin{quote}
\textit{A utilização de uma estratégia híbrida de agrupamento e extração de descritores, entre os 
  graus de pertinência e tipicidade providos pelo método de agrupamento PFCM, permitem o aumento da
    robustez e resiliência contra ruídos na organização flexível de documentos, aumentando assim a
    relevância dos grupos obtidos.}
\end{quote}

Para demonstrar a validade da hipótese formulada, com base na exploração de estratégias existentes
na literatura para o aprimoramento do processo de organização flexível de documentos, definiu-se o seguinte objetivo: 

\begin{quote}
\textit{Conduzir uma investigação em torno dos métodos de agrupamento FCM, PCM e PFCM, para
compreender e interpretar corretamente as peculiaridades de se extrair descritores a partir de um
agrupamento híbrido.}
\end{quote}

A fim de atender a esse objetivo, foram realizadas as seguinte tarefas ao longo do desenvolvimento desta monografia: estudo dos fundamentos teóricos necessários para a organização flexível de documentos, revisão das estratégias recentes
utilizadas por pesquisadores para aprimorar a organização flexível de documentos, condução de diversos experimentos para analisar os impactos da aplicação do algoritmo PFCM no processo de agrupamento e na extração de descritores.

Considerando-se os resultados dos experimentos realizados, foi descoberto que as alterações existentes no PCM, impactam diretamente na qualidade dos descritores extraídos pelo método SoftO-FDCL. Essa descoberta motivou a proposição de dois novos métodos de extração de descritores: {\it Possibilistic Descriptor Comes Last\/} (PDCL) e {\it Mixed - Possibilistic Fuzzy Descriptor Comes Last\/} (Mixed-PFDCL). Os quais apresentaram resultados que
contribuem de maneira significativa para o estado da arte da extração de descritores dos grupos
fuzzy e para o aprimoramento da organização flexível de documentos.

Detalhes de cada tarefa realizada, bem como a comprovação da hipótese levantada, são apresentados ao longo deste trabalho como segue:

{\bf Capítulo \ref{ch:theory}\/}: Neste capítulo, os fundamentos teóricos necessários para compreender o processo de organização flexível de documentos são apresentados, os quais contemplam a descrição da
etapa de pré-processamento e estruturação dos dados; a definição dos principais algoritmos de
agrupamento, capazes de proporcionar flexibilidade ao processo de organização de documentos; e a descrição da extração de
descritores.

{\bf Capítulo \ref{ch:related}\/}: Este capítulo aborda uma breve revisão do estado da arte encontrado
na literatura, referente às diversas estratégias propostas pelos pesquisadores. com o objetivo de
aprimorar todas as etapas da organização flexível de documentos. 

{\bf Capítulo \ref{ch:proposed}\/}: Neste capítulo, um estudo dos impactos da
utilização do método de agrupamento híbrido na organização flexível de documentos é apresentado por meio da realização de análises experimentais. Neste capítulo, as influências das tipicidades presentes
na partição de pertinências do PCM também são apresentadas. Seguidas pela proposta dos métodos PDCL e Mixed-PFDCL. 

{\bf Capítulo \ref{ch:conclusion}\/}: Por fim, este capítulo contempla as conclusões obtidas de todo o estudo
realizado nesta monografia, assim como discussões a respeito dos resultados obtidos nos
experimentos. Aqui também está apontada uma série de possibilidades de extensões que derivam desta
pesquisa. 
