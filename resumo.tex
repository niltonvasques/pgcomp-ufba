O presente trabalho de conclusão de curso intenta realizar, à luz da literatura da área de mineração
de textos e campos do saber correlatos, uma investigação dos impactos dos algoritmos de agrupamento
na composição da organização flexível de documentos textuais. No seu princípio desta monografia,
discute-se a importância e as escolhas utilizadas na etapa pré-processamento, anterior à aplicação
do agrupamento, além dos critérios de validação do agrupamento que são utilizados no etapa do
pós-processamento, a saber, implementado pela medida da silheuta fuzzy. Para contextualizar a
prática da atividade de organização textual flexível proposta, destacam-se ao longo dos capítulos os
desafios inerentes à organização de textos, a exemplo o problema dos altos custos computacionais de
busca de graus de similaridade semântica em matrizes esparsas do tipo atributo valor, assim como os
possíveis mecanismos para mitigar ou reduzir os efeitos negativos dessas dificuldades no processo de
atribuição de significado aos grupos produzidos pelo agrupamento, através da extração de termos
descritores relevantes. A estratégia defendida aqui nesta monografia trata-se de uma abordagem
flexível e híbrida de organização de documentos, mesclando os benefícios concedidos pela adequada
interpretação de partições fuzzy e possibilísticas existentes no método de agrupamento Possibilistic
C-Means e Possibilistic Fuzzy C-Means (PFCM). Como fruto das análises experimentais aqui
desenvolvidas, foram propostos os métodos de extração de descritores Possibilistic Description Comes
Last (PDCL) e Mixed - Possibilistic Fuzzy Description Comes Last (PFDCL). Ambos mostraram-se
relevantes através de evidências experimentais e análises subjetivas à adequação dos métodos, para a
organização flexível de documentos, contribuindo com descobertas originais para o estado da arte da
área. Os resultados da pesquisa ainda estimulam novas implementações cuja execução pode se
transcorrer em trabalhos futuros.
