O presente trabalho de conclusão de curso intenta realizar, à luz da literatura da área de mineração
de textos e campos do saber correlatos, uma investigação dos impactos dos algoritmos de agrupamento
na composição da organização flexível de documentos textuais. Ao longo desta monografia, portanto,
são discutidos aspectos relevantes da etapa de pré-processamento dos documentos a serem organizados; da realização de agrupamento para extração de padrões nestes documentos e dos critérios de validação da organização obtida. Para contextualizar a atividade de organização flexível de documentos proposta, os
desafios inerentes à organização de textos são também destacados a partir da descrição de trabalhos relacionados. A pesquisa realizada para desenvolvimento deste trabalho tem como principal contribuição uma abordagem híbrida para a organização flexível de documentos, mesclando os benefícios concedidos pela adequada
interpretação de partições fuzzy e possibilísticas existentes no método de agrupamento {\it Possibilistic
C-Means} e {\it Possibilistic Fuzzy C-Means} (PFCM). Como fruto das análises aqui
realizadas, foram propostos os métodos de extração de descritores {\it Possibilistic Description Comes
Last} (PDCL) e {\it Mixed - Possibilistic Fuzzy Description Comes Last} (PFDCL). Ambos mostraram-se
relevantes através de evidências experimentais e análises subjetivas à adequação dos métodos, para a
organização flexível de documentos, contribuindo com descobertas originais para o estado da arte. Os resultados da pesquisa ainda estimulam novas implementações cuja execução pode
transcorrer em trabalhos futuros.

%[parágrafo confuso...] Em especial, o problema dos altos custos computacionais do cálculo de similaridade semântica em matrizes do tipo atributo valor, as quais são esparsas e de alta dimensionalidade, são investigados visando propor mecanismos para reduzir os efeitos negativos dessas dificuldades no processo de atribuição de significado aos grupos produzidos pelo agrupamento, através da extração de termos descritores relevantes. 
