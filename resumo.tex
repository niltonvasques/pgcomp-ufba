%A new powerful and flexible organization of documents can be obtained by mixing fuzzy and possibilistic clustering, in which documents can belong to more than one cluster simultaneously with different compatibility degrees with a particular topic. The topics are represented by clusters and the clusters are identified by one or more descriptors extracted by a proposed method. We aim to investigate whether the descriptors extracted after fuzzy and possibilistic clustering improves the flexible organization of documents. Experiments were carried using a collection of documents and we evaluated the descriptors ability to capture the essential information of the used collection. The results prove that the fuzzy possibilistic clusters descriptors extraction is effective and can improve the flexible organization of documents.

Diante da grande quantidade de informações geradas e armazenadas pela humanidade na atualidade, 
vários métodos foram propostos visando processar esses dados. Dentre esses dados, temos uma imensa
quantidade de dados textuais, que por sua vez são não estruturados. Com isso é notória a importância de organizar, de maneira automatizada, esses documentos pelos assuntos ao qual se tratam. Nesse sentido, temos um conjunto de técnicas pertencentes ao campo de estudo da mineração de textos, que visam realizar a tarefa de extrair informações relevantes de documentos textuais. Esta tarefa de análise e extração de informações é 
comumente segmentada nas tarefas de coleta, pré-processamento dos documentos, agrupamento dos dados
e, por fim, a extração de descritores dos grupos obtidos na etapa de agrupamento. Os métodos de agrupamento podem ser separados então pela lógica matemática utilizada, que pode ser a lógica clássica ou a lógica fuzzy. Na lógica clássica, após o agrupamento, cada documento só poderá pertencer a um grupo, enquanto na lógica fuzzy, a pertinência do documento será distribuída entre os grupos. 
Se analisarmos a diversidade de conteúdo em documentos textuais, é trivial notar que frequentemente
um texto aborda um ou mais assunto, o que evidencia a necessidade do desenvolvimento de técnicas para
organizar de maneira flexível os documentos. Para tanto, os métodos de agrupamento fuzzy
tem se mostrado coerentes com a realidade multi temática dos documentos textuais. Por sua vez, o método de agrupamento Fuzzy C-Means (FCM), que é uma adaptação do clássico K-means, se propõe a identificar e separar uma coleção de documentos em grupos, respeitando a lógica multi valorada, permitindo assim que um documento pertença a um ou mais grupos. No entanto, o FCM possui algumas falhas conhecidas, o que motivou a pesquisa e desenvolvimento de métodos alternativos, com o propósito de sanar alguns problemas. Este é o caso dos métodos Possibilístico C-Means (PCM) e Possibilístico Fuzzy C-Means (PFCM). 
Para que os resultados dos diferentes algoritmos de agrupamento e a qualidade da organização flexível de documentos obtida a partir destes algoritmos, é preciso extrair corretamente os descritores dos grupos obtidos, levando em 
consideração a relevância de determinado termo para cada grupo. Com isso temos um cenário no qual é 
preciso combinar métodos de agrupamento fuzzy com métodos de extração de descritores, para obtermos
uma bom resultado no processo de organização dos documentos. A investigação e refinamento dessa 
combinação de métodos, foi a motivação do presente trabalho. Como resultado desse trabalho foi,
proposto extender os experimentos referentes a organização flexível de documentos, utilizando
novos métodos de agrupamento fuzzy existentes na literatura, como o PCM e o PFCM. Assim como também 
foi proposto os métodos de extração de descritores: i) Mixed-PFDCL 
({ \it Mixed - Possibilistic Fuzzy Descriptor Comes Last\/ }), que se utiliza da abordagem híbrida do 
algoritmo PFCM, misturando assim descritores fuzzy e possibilísticos. ii) MixedW-PFDCL
({ \it Mixed Weighted - Possibilistic Fuzzy Descriptor Comes Last\/ }), 
onde além de misturar descritores 
fuzzy e possibilístico, leva em consideração os parâmetros de ponderação do método PFCM.
Além dos métodos de extração de descritores, foi conduzido um estudo dos impactos de se utilizar o 
algoritmo PCM, no método de agrupamento hierárquico HFCM, o que resultou no método de 
agrupamento hierárquico HPCM ({ \it Hierarchical Possibilistic C Means\/ }).
% FALAR SOBRE OS MÈTODOS FUZZY UTILIZADOS, FCM, PFCM, PCM, HFCM, HPCM
% FALAR SOBRE A EXTRAÇÂO DE DESCRITORES
% FALAR SOBRE A MISTURA REALIZADA NA EXTRAÇÃO DE DESCRITORES
