É com muita felicidade que chego neste importante momento na vida de um jovem, o famoso projeto
final. Que além de todo árduo trabalho necessário para a sua conclusão, desempenha um papel
simbólico na carreira de em estudante universitário. Esse evento, é a materialização de toda uma
jornada acadêmica, que começou ainda nos nossos primeiros anos de vida, e que não se conclui aqui.
Ouso a dizer, que para um estudante do interior, oriundo de família humilde, esse momento é ainda
mais marcante. Por isso, não posso deixar
de agradecer a todos os que colaboraram das mais diversas formas durante essa jornada.

Primeiramente gostaria de agradecer a minha mãe Alderita, por ter batalhado a sua vida inteira para
dar melhores condições de vida para seus filhos e ao meu pai Nilton, que se dedicou incansavelmente a
motivar-me e guiar-me nessa jornada. 

A minha irmã Janaina que mesmo distante, se fez presente me lembrando
dos meus deveres como cidadão.

Agradeço a meu amigo Vinícius, por sempre se dispor a me ajudar e também me incentivar. 

A Larissa, por sempre estar ao meu lado, me apoiando nos momentos difíceis. 

A minha orientadora Tatiane, por ter me guiado com bastante dedicação nesse trabalho, sendo sempre
bastante paciente e compreensiva e também por ter dedicado seu tempo nas diversas reuniões que
tivemos.

Ao meu tio Edinaldo, por ter sido ainda desde minha infância uma das pessoas que fez despertar meu
interesse pela computação, pelos estudos e por ter desempenhado um papel fundamental nos meus
primeiros anos aqui em salvador.

Agradeço também ao meu tio Eneildo, por ter sido sempre presente na vida de todos na nossa família e
por também ter me apoiado sempre que precisei.

Ao meu tio Joseni, por ser um exemplo de caráter a ser seguido dentro da família e a mais próxima
demonstração de que o caminho ético é o melhor a seguir. 

Aos todos os meus amigos, pelas alegrias, pela amizade verdadeira e por todos os momentos de
descontração que entre um parágrafo e outro colaboraram para revitalizar as energias.

Aos colegas e amigos do GRUPRO, pelos valiosos momentos de troca e aprimoramento de conhecimentos em
algoritmos durante os treinos promovidos pelo professor Maurício, que tem motivado toda uma nova
geração de estudantes na universidade a se engajar mais na computação.

A todos os professores que tive a oportunidade de adquirir conhecimento ao longo do curso.

Ao Curso de Ciência da Computação, e às pessoas com quem convivi nesses espaços ao longo desses
anos. A experiência de aprendizagem compartilhada entre amigos nos espaços da universidade foram
fundamentais para minha formação acadêmica.
